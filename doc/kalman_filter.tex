\documentclass[]{article}
\usepackage{amsmath}

%opening
\title{"State Estimation with Kalman Filter"}
\author{fjctp (snchan20@yahoo.com)}

\begin{document}

\maketitle

\begin{abstract}
Estimate Euler angles using extended Kalman filter and IMU measurements
\end{abstract}

\section{Equations}
Kalman filter involves two steps: prediction and update.

\subsection{Prediction}
Predict states ($x$) using \textbf{discrete-time model} and update covariance ($P$). Note that, $Q$ is process covariance ($\sigma^2$). In other word, how good is the model. 

\begin{equation}
x' = Fx + Bu\label{eqn:model}
\end{equation}


\begin{equation}
P = FPF^T + Q\label{eqn:predict_covarance}
\end{equation}

\subsection{Update}
Update states ($x$) and covariance ($P$) based on predicted state ($x'$), sensor measurements ($z$), and measurement matrix ($H$). Note that, $R$ is measurement covariance ($\sigma^2$), which is provided by sensor manufacturer or by experiment.

\begin{equation}
z_{est} = Hx'\label{eqn:predict_measurement}
\end{equation}

\begin{equation}
y = z - z_{est}\label{eqn:measurement_difference}
\end{equation}

\begin{equation}
S = HPH^T + R\label{eqn:innovation_covariance}
\end{equation}

\begin{equation}
K = PH^TS^{-1}\label{eqn:kalman_gain}
\end{equation}

\begin{equation}
x = x' + Ky\label{eqn:updated_states}
\end{equation}

\begin{equation}
P = (I-KH)P\label{eqn:updated_covariance}
\end{equation}

\subsection{Extended Kalman Filter}
Extended Kalman filter is similar to Kalman filter. However, the nonlinear model is used for prediction and its Jacobian matrix is calculated for $F$ and $H$.

\paragraph*{}
Eq \eqref{eqn:model} becomes
\begin{equation}
x' = f(x, u)
\end{equation}

\paragraph*{}
$F$ and $H$ in Eq \eqref{eqn:predict_covarance}, \eqref{eqn:innovation_covariance}, \eqref{eqn:kalman_gain}, and \eqref{eqn:updated_covariance} becomes Jacobian matrices that are evaluated at $x'$.

\section{Implementation}
The filter estimates Euler angles using body angular rate ($p, q, r$) and body acceleration ($a_x, a_y, a_z$). This section provides the derivation of the nonlinear model, Jacobian matrix, and initialization of the filter.

\subsection{Model}
The model is based on kinematic that relates Euler angles, body angular rate, and body acceleration. This section begins with frame rotation, and then the nonlinear kinematic equations. And it will ends with the Jacobian matrices. 

\subsubsection{Frame rotation}

\paragraph*{}
Rotate acceleration vector from inertia to body frame

\begin{equation}
\begin{bmatrix}
\dot{u} \\
\dot{v}\\
\dot{w}
\end{bmatrix} = 
R_{i}^{b}
\begin{bmatrix}
a_x \\
a_y \\
a_z
\end{bmatrix}
\end{equation}

where 

\begin{equation}
R_{i}^b = 
\begin{bmatrix}
c_\theta c_\psi & c_\theta s_\psi & -s_\theta\\
(-c_\phi s_\psi + s_\phi s_\theta c_\psi) & (c_\phi c_\psi + s_\phi s_\theta s_\psi) & s_\phi c_\theta\\
(s_\phi s_\psi + c_\phi s_\theta c_\psi) & (-s_\phi c_\psi + c_\phi s_\theta s_\psi) & c_\phi c_\theta\\
\end{bmatrix}
\end{equation}

\paragraph*{}
Rotate body angular rate vector to Euler frame

\begin{equation}
\begin{bmatrix}
\dot{\phi} \\
\dot{\theta}\\
\dot{\psi}
\end{bmatrix} = 
R_{b}^{e}
\begin{bmatrix}
p \\
q \\
r
\end{bmatrix}
\end{equation}

where 

\begin{equation}
R_{b}^e = 
\begin{bmatrix}
1 & s_\phi t_\theta & c_\phi t_\theta\\
0 & c_\phi & -s_\phi\\
0 & s_\phi/c_\theta & c_\phi/c_\theta\\
\end{bmatrix}
\end{equation}

\subsubsection{Kinematic}
Given the velocity, $v$, of an object over time, the position, $p$, becomes

\begin{equation}
p = \int_0^t v dt
\end{equation}

In discrete-time, it is represented as

\begin{equation}
p' = p + v \Delta T
\end{equation}

where $\Delta T$ is the time step.


\paragraph*{Discrete-time model}
By combining the equations above, the equations become:

\begin{equation}
	\begin{bmatrix}
		\phi' \\
		\theta' \\
		\psi'
	\end{bmatrix} = I_{3x3} 
	\begin{bmatrix}
		\phi \\
		\theta \\
		\psi
	\end{bmatrix} + 
	R_b^e(\phi, \theta, \psi)
	\Delta T
	\begin{bmatrix}
		p \\
		q \\
		r
	\end{bmatrix}
\end{equation}

\paragraph{Accelerometer Measurement}
Express measure in terms of filter's states

\begin{equation}
	\begin{bmatrix}
		\dot{u} \\
		\dot{v} \\
		\dot{w} \\
	\end{bmatrix} = 
	R_{i}^{b}(\phi, \theta, \psi)
	\begin{bmatrix}
		0 \\
		0 \\
		g
	\end{bmatrix} = g
	\begin{bmatrix}
		-s_\theta \\
		s_\phi c_\theta \\
		c_\phi c_\theta
	\end{bmatrix}\label{eqn:accelerometer_measurement_equation}
\end{equation}

where $g$ is gravitational acceleration

\subsection{Summary}

We want to estimate Euler angles, $\phi$, $\theta$, and $\psi$ using Extended Kalman filter. $u$ is body angular rate, $p$, $q$, and $r$. $x$ is Euler angles, $\phi$, $\theta$, and $\psi$. $z$ is body acceleration.

\paragraph*{Prediction}

\begin{equation}
	\begin{bmatrix}
		\phi' \\
		\theta' \\
		\psi'
	\end{bmatrix} = I_{3x3} 
	\begin{bmatrix}
		\phi \\
		\theta \\
		\psi
	\end{bmatrix} + 
	R_b^e(\phi, \theta, \psi)
	\Delta T
	\begin{bmatrix}
		p \\
		q \\
		r
	\end{bmatrix}
\end{equation}

where $\Delta T$ is sample time.

\begin{equation}
	P = F_j P F_j^T + Q\label{eqn:predict_covarance}
\end{equation}

where $F_j = I_{3x3}$


\paragraph*{Update}


\begin{equation}
	z_{est} = H_j x'\label{eqn:predict_measurement}
\end{equation}

where measurement matrix, $H_j$, is a Jacobian matrix.

\begin{equation}
	H_j = 
	\begin{bmatrix}
		\frac{\partial \dot{u}}{\partial \phi} && \frac{\partial \dot{u}}{\partial \theta} && \frac{\partial \dot{u}}{\partial \psi} \\
		\frac{\partial \dot{v}}{\partial \phi} && \frac{\partial \dot{v}}{\partial \theta} && \frac{\partial \dot{v}}{\partial \psi} \\
		\frac{\partial \dot{w}}{\partial \phi} && \frac{\partial \dot{w}}{\partial \theta} && \frac{\partial \dot{w}}{\partial \psi}
	\end{bmatrix} = g
	\begin{bmatrix}
		0 && -c_{\theta} && 0 \\
		c_{\phi}c_{\theta} && -s_{\phi}s_{\theta} && 0 \\
		-s_{\phi}c_{\theta} && -c_{\phi}s_{\theta} && 0 \\
	\end{bmatrix}
\end{equation}

\begin{equation}
	y = z - z_{est}\label{eqn:measurement_difference}
\end{equation}

\begin{equation}
	S = H_j P H_j^T + R\label{eqn:innovation_covariance}
\end{equation}

\begin{equation}
	K = P H_j^T S^{-1}\label{eqn:kalman_gain}
\end{equation}

\begin{equation}
	x = x' + Ky\label{eqn:updated_states}
\end{equation}

\begin{equation}
	P = (I-KH)P\label{eqn:updated_covariance}
\end{equation}


\subsection{Initialization}
Initialize filter's states, $\phi, \theta, \psi$ using accelerometer measurement, which is body acceleration, $a_{bx}, a_{by}, a_{bz}$.

Using Equation \ref{eqn:accelerometer_measurement_equation}, we have 
\begin{equation}
\begin{bmatrix}
a_{bx} \\
a_{by} \\
a_{bz} \\
\end{bmatrix} = g
\begin{bmatrix}
-s_\theta \\
s_\phi c_\theta \\
c_\phi c_\theta
\end{bmatrix}
\end{equation}

We can express $\phi$ and $\theta$ as
\begin{equation}
\theta = -asin (\frac{a_{bx}}{g})\label{eqn:initialization:theta}
\end{equation}

\begin{equation}
\phi = asin (\frac{a_{by}}{g c_{\theta}})\label{eqn:initialization:phi}
\end{equation}

Since there is no magnetometer, we assume
\begin{equation}
\psi = 0
\end{equation}

Assume that the accelerometer is at rest, it returns gravitational acceleration only.

\begin{equation}
g = \sqrt{a_{bx}^2 + a_{by}^2 + a_{bz}^2}
\end{equation} 


\end{document}
